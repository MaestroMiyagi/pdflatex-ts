\documentclass[12pt]{article}
\usepackage[utf8]{inputenc}
\usepackage[spanish]{babel}
\usepackage{geometry}
\usepackage{amsmath}
\usepackage{graphicx}
\geometry{margin=2.5cm}

\title{Documento de Ejemplo para pdflatex-ts}
\author{Sistema de Pruebas}
\date{\today}

\begin{document}

\maketitle

\tableofcontents
\newpage

\section{Introducción}

Este es un documento de ejemplo para probar la funcionalidad de la librería \texttt{pdflatex-ts}. 
La librería permite convertir archivos LaTeX a PDF de manera programática usando TypeScript.

\section{Características Principales}

\subsection{Conversión de Archivos}
La librería puede convertir archivos \texttt{.tex} existentes a formato PDF:

\begin{itemize}
    \item Soporte para documentos complejos
    \item Manejo de errores robusto
    \item Configuración flexible de opciones
    \item Limpieza automática de archivos auxiliares
\end{itemize}

\subsection{Generación Dinámica}
También permite generar PDFs dinámicamente desde contenido LaTeX en memoria:

\begin{enumerate}
    \item Crear contenido LaTeX programáticamente
    \item Convertir a PDF sin archivos temporales visibles
    \item Integrar con sistemas de reportes automáticos
\end{enumerate}

\section{Ejemplo Matemático}

La librería maneja correctamente documentos con contenido matemático:

\begin{equation}
    E = mc^2
\end{equation}

\begin{align}
    \int_{-\infty}^{\infty} e^{-x^2} dx &= \sqrt{\pi} \\
    \sum_{n=1}^{\infty} \frac{1}{n^2} &= \frac{\pi^2}{6}
\end{align}

\section{Tablas y Contenido Estructurado}

\begin{table}[h]
\centering
\begin{tabular}{|l|c|r|}
\hline
\textbf{Característica} & \textbf{Soporte} & \textbf{Versión} \\
\hline
Callbacks & ✓ & 1.0.0 \\
Promises/Async & ✓ & 1.0.0 \\
Generación Dinámica & ✓ & 1.0.0 \\
TypeScript & ✓ & 1.0.0 \\
\hline
\end{tabular}
\caption{Características soportadas por pdflatex-ts}
\end{table}

\section{Código de Ejemplo}

A continuación se muestra un ejemplo de uso básico:

\begin{verbatim}
import { LatexToPdfConverter } from 'pdflatex-ts'

const converter = new LatexToPdfConverter()

const result = await converter.convertAsync('input.tex', {
  output: 'output/documento.pdf',
  debug: true
})

console.log('PDF generado:', result.outputPath)
\end{verbatim}

\section{Conclusión}

La librería \texttt{pdflatex-ts} proporciona una interfaz moderna y fácil de usar para 
la conversión de documentos LaTeX a PDF en entornos Node.js y TypeScript.

Sus principales ventajas son:
\begin{itemize}
    \item API intuitiva y bien documentada
    \item Soporte completo para TypeScript
    \item Manejo robusto de errores
    \item Flexibilidad en la configuración
\end{itemize}

\end{document}
